\documentclass[11pt]{article}
\usepackage{stmaryrd, amsmath,amssymb,amsfonts,amsthm}
\newcommand{\numpy}{{\tt numpy}}    % tt font for numpy
\usepackage{graphicx}

\topmargin -.5in
\textheight 9in
\oddsidemargin -.25in
\evensidemargin -.25in
\textwidth 7in

\begin{document}


\section{LP page 142, Exercise 5.1.1}\label{prob:1}

Here we list all possible initial bfs for the given problem.

\begin{enumerate}
    \item \begin{align*}
        & B = [A_1, A_2]=\begin{bmatrix} 1 & 3 \\ 0 & 1\end{bmatrix} & N = [A_3, A_4]=\begin{bmatrix} 2 & 2 \\ 0 & 1\end{bmatrix} \\
        & x_B = \begin{bmatrix} 1 & -3 \\ 0 & 1\end{bmatrix} \begin{bmatrix} 5 \\ 6 \end{bmatrix} = \begin{bmatrix} -13 \\ 6 \end{bmatrix} & x_N = \begin{bmatrix} 0 \\ 0\end{bmatrix} \\
        & x = (-13, 6, 0, 0) &\\
        & \text{Non-negativity constraint violated. No bfs generated.} &
    \end{align*}
    
    \item \begin{align*}
        & B = [A_1, A_3]=\begin{bmatrix} 1 & 2 \\ 0 & 2\end{bmatrix} & N = [A_2, A_4]=\begin{bmatrix} 3 & 0 \\ 1 & 1\end{bmatrix} \\
        & x_B = \begin{bmatrix} 1 & -1 \\ 0 & \frac{1}{2}\end{bmatrix} \begin{bmatrix} 5 \\ 6 \end{bmatrix} = \begin{bmatrix} -1 \\ 3 \end{bmatrix} & x_N = \begin{bmatrix} 0 \\ 0\end{bmatrix} \\
        & x = (-1, 0, 3, 0) &\\
        & \text{Non-negativity constraint violated. No bfs generated.} &
    \end{align*}
    
    \item \begin{align*}
        & B = [A_1, A_4] = \begin{bmatrix} 1 & 0 \\ 0 & 1\end{bmatrix}& N = [A_2, A_3] =\begin{bmatrix} 3 & 2 \\ 1 & 2\end{bmatrix}\\    
        & x_B = \begin{bmatrix} 1 & 0 \\ 0 & 1\end{bmatrix} \begin{bmatrix} 5 \\ 6 \end{bmatrix} = \begin{bmatrix} 5 \\ 6 \end{bmatrix} & x_N = \begin{bmatrix} 0 \\ 0\end{bmatrix} \\
        & x = (5, 0, 0, 6) & \\
        & \text{Objective function yield a value -11} &
    \end{align*}
    
    \item \begin{align*}
        & B = [A_2, A_3]=\begin{bmatrix} 3 & 2 \\ 1 & 2\end{bmatrix} & N = [A_1, A_4]= \begin{bmatrix} 1 & 0 \\ 0 & 1\end{bmatrix}\\
        & x_B = \begin{bmatrix} \frac{1}{2} & -\frac{1}{2} \\ -\frac{1}{4} & \frac{3}{4}\end{bmatrix} \begin{bmatrix} 5 \\ 6 \end{bmatrix} = \begin{bmatrix}  -\frac{1}{2}\\ \frac{13}{4} \end{bmatrix} & x_N = \begin{bmatrix} 0 \\ 0\end{bmatrix} \\
        & x = (0, -\frac{1}{2}, \frac{13}{4}, 0) &\\
        & \text{Non-negativity constraint violated. No bfs generated.} &
    \end{align*}
    
    \item \begin{align*}
        & B = [A_2, A_4]=\begin{bmatrix} 3 & 0 \\ 1 & 1\end{bmatrix} & N = [A_1, A_3]=\begin{bmatrix} 1 & 2 \\ 0 & 2\end{bmatrix}\\
        & x_B = \begin{bmatrix} \frac{1}{3} &  0 \\ -\frac{1}{3} & 1 \end{bmatrix} \begin{bmatrix} 5 \\ 6 \end{bmatrix} = \begin{bmatrix}  \frac{5}{3} \\ \frac{13}{3} \end{bmatrix} & x_N = \begin{bmatrix} 0 \\ 0\end{bmatrix} \\
        & x = (0, \frac{5}{3}, 0,\frac{13}{3}) &\\
        & \text{Objective function yield a value -1} &
    \end{align*}
    
    \item \begin{align*}
        & B = [A_3, A_4]=\begin{bmatrix} 2 & 0 \\ 2 & 1\end{bmatrix}& N = [A_1, A_2]=\begin{bmatrix} 1 & 3 \\ 0 & 1\end{bmatrix} \\
        & x_B = \begin{bmatrix} \frac{1}{2} & 0 \\ -1 & 1 \end{bmatrix} \begin{bmatrix} 5 \\ 6 \end{bmatrix} = \begin{bmatrix} \frac{5}{2} \\ 1 \end{bmatrix} & x_N = \begin{bmatrix} 0 \\ 0\end{bmatrix} \\
        & x = (0, 0, \frac{5}{2}, 1) &\\
        &\text{Objective function yield a value $\frac{3}{2}$} &
    \end{align*}
\end{enumerate}

\section{LP page 142, Exercise 5.1.2}

\begin{enumerate}
    \item Is $x=(0, 0, 3, 0, 2)$ an bfs? \par
    
    \begin{align*}
            & B = [A_3, A_5] = \begin{bmatrix}1 & 0 \\ 0 & 1\end{bmatrix} & N = [A_1, A_2, A_4] = \begin{bmatrix} 4 & 2 & 1 \\ 2 & 2 & 3 \end{bmatrix}
    \end{align*}
    Since $B$ is the identity matrix, $B$ is invertible and $$B^{-1} = B$$
    Moreover, 
    \begin{enumerate}
        \raggedright 
        \item $ x_B = B^{-1}b = I_2 \begin{bmatrix} 3 \\  2\end{bmatrix} = \begin{bmatrix} 3 \\ 2\end{bmatrix}$
        \item $x_B \geq 0$
        \item $x_N = \begin{bmatrix} 0 \\ 0 \\ 0 \end{bmatrix}$
    \end{enumerate}
    Thus, $x=(0, 0, 3, 0, 2)$ is an bfs for Exercise 5.1.2

    \item Is $x=(1, 0, -1, 0, 0)$ an bfs? \par
    It is obvious that $$x_B = [x_1, x_3] = \begin{bmatrix} 1 \\ -1\end{bmatrix} \leq 0$$
    
    Thus, $x=(1, 0, -1, 0, 0)$ is not an bfs for Exercise 5.1.2
    
    \item Is $x=(0, 1, 1, 0, 0)$ an bfs? \par
    \begin{align*}
        & B = [A_2, A_3] = \begin{bmatrix}2 & 1 \\ 2 & 0\end{bmatrix} & N = [A_1, A_4, A_5] = \begin{bmatrix} 4 & 1 & 0 \\ 2 & 3 & 1 \\ \end{bmatrix}
    \end{align*}
    The inverse of B is $$B^{-1} = \begin{bmatrix} 0 &  \frac{1}{2} \\ 1 & -1 \end{bmatrix}$$
    $x_B = B^{-1}b = \begin{bmatrix} 0 &  \frac{1}{2} \\ 1 & -1 \end{bmatrix} \begin{bmatrix} 3 \\  2\end{bmatrix} = \begin{bmatrix} 1 \\ 1\end{bmatrix} $
    
    Thus, $x=(0, 1, 1, 0, 0)$ is an bfs for Exercise 5.1.2
    
    \item Is $x=(\frac{1}{2}, 1, 0, 0, 0)$ an bfs? \par
    \begin{align*}
        & B = [A_1, A_2] = \begin{bmatrix}4 & 2 \\ 2 & 2\end{bmatrix} & N = [A_3, A_4, A_5] = \begin{bmatrix} 1 & 1 & 0 \\ 0 & 3 & 1 \end{bmatrix}
    \end{align*}
    The inverse of B is $$B^{-1} = \begin{bmatrix} 0.5 & -0.5 \\ -0.5 & 1\end{bmatrix}$$
    
    However,
    $x_B = B^{-1}b = \begin{bmatrix} 0.5 & -0.5 \\ -0.5 & 1\end{bmatrix} \begin{bmatrix} 3 \\  2\end{bmatrix} = \begin{bmatrix} 0.5 \\ 0.5\end{bmatrix} \not= \begin{bmatrix} \frac{1}{2} \\ 1\end{bmatrix}$
    
    Thus, $x=(\frac{1}{2}, 1, 0, 0, 0)$ is not an bfs for Exercise 5.1.2
\end{enumerate}

\section{LP page 142, Exercise 5.1.3}
$x=(2,3,1,0,1,0, 4)$ is not a bfs for the given LP problem. 

The given LP problem has 4 equations and 7 variables, among which 4 are basic variables, and 3 are non-basic ones. For non-basic variables, their corresponding value should be zero. Therefore, we expect to see at least 3 zeros for any bfs, which is not shown in the given solution. Therefore, the given solution $x=(2,3,1,0,1,0, 4)$ is not a bfs for the given LP problem. 

\section{LP page 145, Exercise 5.2.1}
Here we perform optimally test using $c_N - c_B B^{-1}N$ for each bfs in Problem \ref{prob:1}.

\begin{enumerate}
    \item bfs 3\par
    \begin{align*}
        & B = [A_1, A_4] = \begin{bmatrix}1 & 0 \\ 0 & 1\end{bmatrix} & N = [A_2, A_3] = \begin{bmatrix}3 & 2 \\ 1 & 2\end{bmatrix}\\
        & B^{-1} = \begin{bmatrix} 1 & 0 \\ 0 & 1\end{bmatrix} & \\
        & c_B = (-1, -1) & c_N =(2, 1)\\
    \end{align*}
    \begin{align*}
         c_N - c_B B^{-1}N &= \begin{bmatrix} 2 \\ 1 \end{bmatrix} - \begin{bmatrix} -1 \\ -1 \end{bmatrix} \begin{bmatrix} 1 & 0 \\ 0 & 1\end{bmatrix} \begin{bmatrix}3 & 2 \\ 1 & 2\end{bmatrix} \\
        &= \begin{bmatrix} 2 \\ 1 \end{bmatrix} - \begin{bmatrix} -4 \\ -4 \end{bmatrix} \\
        &= \begin{bmatrix} 6 \\ 5 \end{bmatrix}
    \end{align*}
    
    \item bfs 5\par
    \begin{align*}
        & B = [A_2, A_4]=\begin{bmatrix} 3 & 0 \\ 1 & 1\end{bmatrix} & N = [A_1, A_3]=\begin{bmatrix} 1 & 2 \\ 0 & 2\end{bmatrix}\\
        & B^{-1} = \begin{bmatrix} \frac{1}{3} &  0 \\ -\frac{1}{3} & 1 \end{bmatrix} & \\
        & c_B = (2, -1) & c_N =(-1,1)\\
    \end{align*}
    \begin{align*}
         c_N - c_B B^{-1}N &=  \begin{bmatrix} -1 \\ 1 \end{bmatrix}  - \begin{bmatrix} 2 \\ -1 \end{bmatrix} \begin{bmatrix} \frac{1}{3} &  0 \\ -\frac{1}{3} & 1 \end{bmatrix} \begin{bmatrix} 1 & 2 \\ 0 & 2\end{bmatrix}\\
        &= \begin{bmatrix} -2 \\ 1 \end{bmatrix}
    \end{align*}
    
    \item bfs 6\par
    \begin{align*}
        & B = [A_3, A_4]=\begin{bmatrix} 2 & 0 \\ 2 & 1\end{bmatrix}& N = [A_1, A_2]=\begin{bmatrix} 1 & 3 \\ 0 & 1\end{bmatrix} \\
        & B^{-1} = \begin{bmatrix} \frac{1}{2} & 0 \\ -1 & 1 \end{bmatrix} & \\
        & c_B = (1, -1) & c_N =(-1,2)\\
    \end{align*}
    \begin{align*}
         c_N - c_B B^{-1}N &=  \begin{bmatrix} -1 \\ 2 \end{bmatrix}  - \begin{bmatrix} 1 \\ -1 \end{bmatrix} \begin{bmatrix} \frac{1}{2} & 0 \\ -1 & 1 \end{bmatrix} \begin{bmatrix} 1 & 3 \\ 0 & 1\end{bmatrix}\\
        &= \begin{bmatrix} -2.5 \\ -1.5\end{bmatrix}
    \end{align*}
\end{enumerate}

From proposition 5.4, above optimality test shows that bfs 3 to be the optimal solution for the LP problem.

\section{LP page 145, Exercise 5.2.3}

The test for LP with maximization will be $c_N - c_B B^{-1}N \leq 0$. The justification is that the maximization of a function could be easily re-written as minimization of the negation of the function. 

\textit{proof not required, omitted.}

\section{LP page 150, Exercise 5.3.2}
\begin{align*}
    & c = (2, -3, -1, 4, 1) & d =(0, 1, 3, 1, 2) \\
    & cd = 0
\end{align*}
From Proposition 5.5, since $cd = 0$, moving in the direction $ d =(0, 1, 3, 1, 2)$ will not improve the objective function.

\section{LP page 150, Exercise 5.3.3}

\begin{align*}
    & B = [A_4, A_5] = \begin{bmatrix} 1 & 0 \\ 0 & 1\end{bmatrix} & N = [A_3, A_2, A_1] = \begin{bmatrix} 3 & 3 & 2 \\ -1 & 4 & 3\end{bmatrix} \\
    & B^{-1} = B \\
    & c_B = (1, 1) & c_N = (-4, -1, 2) \\
\end{align*}
\begin{align*}
    c_N - c_B B^{-1}N &= \begin{bmatrix} -4 \\ -1 \\ 2 \end{bmatrix}  - \begin{bmatrix} 1 \\ 1 \end{bmatrix} \begin{bmatrix} 1 & 0 \\ 0 & 1 \end{bmatrix} \begin{bmatrix} 3 & 3 & 2 \\ -1 & 4 & 3\end{bmatrix} \\
    &= \begin{bmatrix} -6 \\ -8 \\ -3 \end{bmatrix}
\end{align*}

Now, we determine the direction of movement by the three method, respectively.

\begin{enumerate}
    \item First come, first serve.\par
    This method will yield $j=1$.
    \begin{align*}
    & d_B = -B^{-1}N_{\cdot 1} =-\begin{bmatrix} 1 & 0 \\ 0 & 1\end{bmatrix}\begin{bmatrix} 3 \\ -1 \end{bmatrix} = \begin{bmatrix} -3 \\ 1 \end{bmatrix} \\
    & d_N = I_{3 \cdot 1} = \begin{bmatrix} 1 \\ 0 \\ 0 \end{bmatrix}
    \end{align*}
    Thus, the direction of movement is $$d = (d_B, d_N) = (0, 0, 1, -3, 1)$$
    \item Bland's rule.\par
    This method will yield $j=3$, since $A_1$ is in column 3. 
    \begin{align*}
    & d_B = -B^{-1}N_{\cdot 3} =-\begin{bmatrix} 1 & 0 \\ 0 & 1\end{bmatrix}\begin{bmatrix} 2 \\ 3 \end{bmatrix} = \begin{bmatrix} -2 \\ -3 \end{bmatrix} \\
    & d_N = I_{3 \cdot 3} = \begin{bmatrix} 0 \\ 0 \\ 1 \end{bmatrix}
    \end{align*}
    Thus, the direction of movement is $$d = (d_B, d_N) = (1, 0, 0, -2, -3)$$
    \item Steepest descent.\par
    This method will yield $j=2$.
    \begin{align*}
    & d_B = -B^{-1}N_{\cdot 2} =-\begin{bmatrix} 1 & 0 \\ 0 & 1\end{bmatrix}\begin{bmatrix} 3 \\ 4 \end{bmatrix} = \begin{bmatrix} -3 \\ -4 \end{bmatrix} \\
    & d_N = I_{3\cdot 2} = \begin{bmatrix} 0 \\ 1 \\ 0 \end{bmatrix}
    \end{align*}
    Thus, the direction of movement is $$d = (d_B, d_N) = (0, 1, 0, -3, -4)$$
\end{enumerate}

\section{LP page 157, 5.4.3}

Known that $x_B=(x_3,x_2) = (1,2)$ and $d_B=(-\frac{1}{3},-\frac{2}{3})$

Applying the ratio test, $-x_B \varoslash d_B = (3, 3)$,

\begin{enumerate}
    \item Bland's rule\par
    This will yields $k=2$, which results in $t^*=3$
    \item First come, first serve rule\par
    This will yields $k=1$, which results in $t^*=3$
\end{enumerate}

\section{LP page 157, 5.4.5}
\begin{align*}
    & B = [A_4, A_5] = \begin{bmatrix} 1 & 0 \\ 0 & 1\end{bmatrix} & N = [A_1, A_2, A_3] = \begin{bmatrix} -7 & 2 & 1 \\ -8 & 9 & 6\end{bmatrix} \\
    & B^{-1} = B \\
    & x_B = B^{-1}b = I_2 \begin{bmatrix} 25 \\ 35 \end{bmatrix} = \begin{bmatrix} 25 \\ 35 \end{bmatrix} & x_N = \begin{bmatrix} 0 \\ 0 \\ 0 \end{bmatrix} \\
    & x = (0, 0, 0, 25, 35) \\
    & c_B = (0, 0) & c_N = (-7, 6, 4) \\
\end{align*}
\begin{align*}
    c_N - c_B B^{-1}N &= \begin{bmatrix} -7 \\ 6 \\ 4 \end{bmatrix}  - \begin{bmatrix} 0 \\ 0 \end{bmatrix} \begin{bmatrix} 1 & 0 \\ 0 & 1 \end{bmatrix} \begin{bmatrix} -7 & 2 & 1 \\ -8 & 9 & 6\end{bmatrix} \\
    &= \begin{bmatrix} -7 \\ 6 \\ 4 \end{bmatrix}
\end{align*}
Following the steepest descent rule, we have $j = 1$.
\begin{align*}
& d_B = -B^{-1}N_{\cdot 1} = -\begin{bmatrix} 1 & 0 \\ 0 & 1\end{bmatrix}\begin{bmatrix} -7 \\ -8 \end{bmatrix} = \begin{bmatrix} 7 \\ 8 \end{bmatrix} \\
& d_N = I_{1\cdot 3} = \begin{bmatrix} 1 \\ 0 \\ 0 \end{bmatrix}
\end{align*}
Since $d_B = \begin{bmatrix} 7 \\ 8 \end{bmatrix}  \geq 0 $, $t^*=\infty$

\section{LP page 164, 5.5.1}
Known that $x=(0, 2, 0, 0, 3, 0)$,$j^*=4$, $d=(0, -1, 0, 0, 1, 1)$, $t^*=2$, $k^*=1$
\begin{align*}
    & B = [A_2, A_5] & N = [A_1, A_3, A_4, A_6] \\
\end{align*}

By swapping column $k^*=1$ of B and column $j^*=4$ of N,
\begin{align*}
    & B' = [A_6, A_5] & N' = [A_1, A_3, A_4, A_2]\\
\end{align*}

The new basic variables are $x_B = (x_6, x_5)$, non-basic variables are $x_N = (x_1, x_3, x_4, x_2)$.

\subsection{What are the new B and N matrices?}
\begin{align*}
    & B = [A_2, A_5] = \begin{bmatrix} 1 & 0 \\ 0 & 1\end{bmatrix} & N = [A_1, A_3, A_4, A_6] = \begin{bmatrix} 1 & -2 & -1 & 1 \\ 2 & 1 & 2 & -1\end{bmatrix} \\
\end{align*}
By swapping column $k^*=1$ of B and column $j^*=4$ of N,
\begin{align*}
    & B' = [A_6, A_5] = \begin{bmatrix} 1 & 0 \\ -1 & 1\end{bmatrix} & N' = [A_1, A_3, A_4, A_2] = \begin{bmatrix} 1 & -2 & -1 & 1 \\ 2 & 1 & 2 & 0\end{bmatrix} \\
\end{align*}

\subsection{Compute the new inverse matrix.}
$$B'^{-1} = \begin{bmatrix} 1 & 0 \\ 1 & 1\end{bmatrix}$$

\subsection{Compute the new basic variables by using the new inverse matrix.}
$$x_{B'} = B'^{-1} b = \begin{bmatrix} 1 & 0 \\ 1 & 1\end{bmatrix} \begin{bmatrix} 2 \\ 3\end{bmatrix} = \begin{bmatrix} 2 \\ 5\end{bmatrix}$$

\subsection{Compute the new basic variables by using the $x, t^*, d$.}
$$x' = x + t^*d = (0, 2, 0, 0, 3, 0) + 2 \cdot (0, -1, 0, 0, 1, 1) = (0, 0, 0, 0, 5, 2)$$

\subsection{Test the new bfs for optimality.}
$$c_B = (1, -4), c_N = (1, 1, 0, -2)$$
\begin{align*}
    c_N - c_B B^{-1} N &= \begin{bmatrix} 1 \\ 1 \\ 0 \\ -2\end{bmatrix} - \begin{bmatrix} 1 \\ -4\end{bmatrix} \begin{bmatrix} 1 & 0 \\ 1 & 1\end{bmatrix} \begin{bmatrix} 1 & -2 & -1 & 1 \\ 2 & 1 & 2 & 0\end{bmatrix} \\
    &=\begin{bmatrix} 12 \\ -1 \\ 5 \\ 1\end{bmatrix}
\end{align*}
Not an optimal solution.

\end{document}